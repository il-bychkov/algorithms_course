%!TEX TS-program = xelatex

% HSE Beamer Theme
% by Danil Fedorovykh
% http://hse.ru/staff/df
%
% Version 2.0 (English)
% January 2022

%%% Set up the free HSE Sans font
%%% https://www.hse.ru/info/brandbook/#font

\documentclass[aspectratio=169]{beamer}

\newbool{russian}
\booltrue{russian} % Uncomment if in Russian
\usepackage{HSE-theme/beamerthemeHSE} % Load HSE theme
\usepackage[no-math]{fontspec}      % fonts loading
\usepackage{caption}
\usepackage{subfigure}
\usepackage{subcaption}
\usepackage{hyperref}
\usepackage[dvipsnames]{xcolor}
\usepackage{ragged2e}
\captionsetup[figure]{labelformat=empty}
\captionsetup[subfigure]{labelformat=empty}
\renewcommand*{\thesubfigure}{(\arabic{subfigure})}
\setsansfont{HSE Sans}
%\graphicspath{{./images/}}
\graphicspath{{/home/llyy/Yandex.Disk/personal/knowledge/general/algorithms_course/repo/algorithms_course/5_simple_data_structures/lection/images}}


%%% Информация об авторе и выступлении
\title[Title]{Алгоритмы и структуры данных} 
\subtitle{Лекция 5. Простые структуры данных}
\author[Author's name]{Илья Сергеевич Бычков\\ \smallskip \scriptsize \url{ibychkov@hse.ru}}
\institute{НИУ ВШЭ - Нижний Новгород}
\date{\today}


\begin{document}

\frame[plain]{\titlepage}

%%%%%%%%%%%%%%%%%%%%%%%%%%%%%%%%%%%%%%%%%%%%%%%%%%%%%%%%%%%%%%%%%%%%%%%%%%%%%%%%%%%%%%%%%%%%%%%%%%
\begin{frame}[c]
%\frametitle{A first slide}

\begin{center}
\Huge Лекция 5.

\Huge Простые структуры данных
\end{center}

\end{frame}

%%%%%%%%%%%%%%%%%%%%%%%%%%%%%%%%%%%%%%%%%%%%%%%%%%%%%%%%%%%%%%%%%%%%%%%%%%%%%%%%%%%%%%%%%%%%%%%%%%
\begin{frame}
\frametitle{Простые структуры данных}
\framesubtitle{План лекции}

\begin{enumerate}
  \setcounter{enumi}{-1}
  \item{План лекции}
  \item{\textcolor{blue}{Связный список (linked list)}}
  \item{Стек (stack)}
  \item{Очередь (queue)}
\end{enumerate}
\end{frame}



%%%%%%%%%%%%%%%%%%%%%%%%%%%%%%%%%%%%%%%%%%%%%%%%%%%%%%%%%%%%%%%%%%%%%%%%%%%%%%%%%%%%%%%%%%%%%%%%%%
\begin{frame}
\frametitle{Связный список}
\framesubtitle{Связный список}
\justifying
\textcolor{red}{Связный список (linked list)} - это структура данных, состоящая  объектов специального вида, которые называются  \textcolor{blue}{узлами (nodes)}. Узлы хранят сами данные и связаны друг с другом с помощью указателей.\newline\newline
Каждый узел содержит одно или несколько полей для хранения данных\newline\newline
Каждый узел содержит указатель на следующий/предыдущий узел\newline\newline
Требуются указатели на первый/последний элементы списка\newline\newline
\end{frame}

%%%%%%%%%%%%%%%%%%%%%%%%%%%%%%%%%%%%%%%%%%%%%%%%%%%%%%%%%%%%%%%%%%%%%%%%%%%%%%%%%%%%%%%%%%%%%%%%%%
\begin{frame}
\frametitle{Связный список}
\framesubtitle{Структура node}
\justifying
\textcolor{red}{Связный список (linked list)}\newline

\begin{figure}
    \captionsetup[subfigure]{labelformat=empty}
    \centering
    \subfigure[{ \scriptsize Структура данных node, Источник - \href{https://www.learn-c.org/en/Linked_lists}{Ссылка}}]{\includegraphics[width=60mm, height=22mm]{ll_node_code}} \quad\quad\quad
        \subfigure[{ \scriptsize Создание node, Источник - \href{https://www.learn-c.org/en/Linked_lists}{Ссылка}}]{\includegraphics[width=65mm, height=40mm]{ll_node_create}} 
\end{figure}
\end{frame}

%%%%%%%%%%%%%%%%%%%%%%%%%%%%%%%%%%%%%%%%%%%%%%%%%%%%%%%%%%%%%%%%%%%%%%%%%%%%%%%%%%%%%%%%%%%%%%%%%%
\begin{frame}
\frametitle{Связный список}
\framesubtitle{Операции}
Основные операции\newline\newline
\textcolor{blue}{Вставка и удаление}
\begin{itemize}
  \item{В начало/конец списка}
  \item{До/после определенного значения}
  \item{До/после определенного адреса}
\end{itemize}
\textcolor{blue}{Поиск}
\begin{itemize}
  \item{По значению}
  \item{По позиции}
\end{itemize}
\begin{figure}
    \captionsetup[subfigure]{labelformat=empty}
    \centering
        \subfigure[{ \scriptsize Связный список, Источник - \href{https://www.geeksforgeeks.org/what-is-linked-list/}{Geeks4Geeks}}]{\includegraphics[width=80mm, height=20mm]{ll_single}} 
\end{figure}
\end{frame}

%%%%%%%%%%%%%%%%%%%%%%%%%%%%%%%%%%%%%%%%%%%%%%%%%%%%%%%%%%%%%%%%%%%%%%%%%%%%%%%%%%%%%%%%%%%%%%%%%%
\begin{frame}
\frametitle{Связный список}
\framesubtitle{Перебор связного списка}
\justifying
Перебор элементов связного списка

\begin{figure}
    \captionsetup[subfigure]{labelformat=empty}
    \centering
    \subfigure[{ \scriptsize Перебор элементов связного списка, Источник - \href{https://www.learn-c.org/en/Linked_lists}{Ссылка}}]{\includegraphics[width=90mm, height=40mm]{ll_traverse}}
\end{figure}
\end{frame}

%%%%%%%%%%%%%%%%%%%%%%%%%%%%%%%%%%%%%%%%%%%%%%%%%%%%%%%%%%%%%%%%%%%%%%%%%%%%%%%%%%%%%%%%%%%%%%%%%%
\begin{frame}
\frametitle{Связный список}
\framesubtitle{Вариации связных списков}
\justifying
\textcolor{red}{Односвязный список (singly linked list)} - состоит из узлов, которые хранят полезные данные и указатель на следующий

\begin{figure}
    \captionsetup[subfigure]{labelformat=empty}
    \centering
    \subfigure[{ \scriptsize Односвязный список, Источник - \href{https://www.sanfoundry.com/c-program-implement-singly-linked-list/}{Ссылка}}]{\includegraphics[width=90mm, height=40mm]{ll_single_variant}}
\end{figure}
\end{frame}

%%%%%%%%%%%%%%%%%%%%%%%%%%%%%%%%%%%%%%%%%%%%%%%%%%%%%%%%%%%%%%%%%%%%%%%%%%%%%%%%%%%%%%%%%%%%%%%%%%
\begin{frame}
\frametitle{Связный список}
\framesubtitle{Вариации связных списков}
\justifying
\textcolor{red}{Двусвязный список (doubly linked list)} - состоит из узлов, которые хранят полезные данные, указатели на предыдущий узел и следующий узел

\begin{figure}
    \captionsetup[subfigure]{labelformat=empty}
    \centering
    \subfigure[{ \scriptsize Двусвязный список, Источник - \href{https://www.geeksforgeeks.org/why-use-a-doubly-linked-list/}{Geeks4Geeks}}]{\includegraphics[width=130mm, height=40mm]{ll_doubly_variant}}
\end{figure}
\end{frame}

%%%%%%%%%%%%%%%%%%%%%%%%%%%%%%%%%%%%%%%%%%%%%%%%%%%%%%%%%%%%%%%%%%%%%%%%%%%%%%%%%%%%%%%%%%%%%%%%%%
\begin{frame}
\frametitle{Связный список}
\framesubtitle{Вариации связных списков}
\justifying
\textcolor{red}{Кольцевой связный список (circular linked list)} - разновидность связного списка, при которой первый элемент указывает на последний, а последний — на первый
\begin{figure}
    \captionsetup[subfigure]{labelformat=empty}
    \centering
    \subfigure[{ \scriptsize Односвязный список, Источник - \href{https://www.sanfoundry.com/c-program-implement-singly-linked-list/}{Ссылка}}]{\includegraphics[width=120mm, height=40mm]{ll_circular_variant}}
\end{figure}
\end{frame}

%%%%%%%%%%%%%%%%%%%%%%%%%%%%%%%%%%%%%%%%%%%%%%%%%%%%%%%%%%%%%%%%%%%%%%%%%%%%%%%%%%%%%%%%%%%%%%%%%%
\begin{frame}
\frametitle{Связный список}
\framesubtitle{Сложность операций}
\justifying
\textcolor{red}{Односвязный список (singly linked list)}
\begin{figure}
    \captionsetup[subfigure]{labelformat=empty}
    \centering
    \subfigure[{ \scriptsize Односвязный список - сложность, Источник - Этот курс}]{\includegraphics[width=85mm, height=55mm]{ll_singly_complexity_table}}
\end{figure}
\end{frame}

%%%%%%%%%%%%%%%%%%%%%%%%%%%%%%%%%%%%%%%%%%%%%%%%%%%%%%%%%%%%%%%%%%%%%%%%%%%%%%%%%%%%%%%%%%%%%%%%%%
\begin{frame}
\frametitle{Связный список}
\framesubtitle{Сложность операций}
\justifying
\textcolor{red}{Двусвязный список (doubly linked list)}
\begin{figure}
    \captionsetup[subfigure]{labelformat=empty}
    \centering
    \subfigure[{ \scriptsize Двусвязный список - сложность, Источник - Этот курс}]{\includegraphics[width=80mm, height=55mm]{ll_doubly_complexity_table}}
\end{figure}
\end{frame}

%%%%%%%%%%%%%%%%%%%%%%%%%%%%%%%%%%%%%%%%%%%%%%%%%%%%%%%%%%%%%%%%%%%%%%%%%%%%%%%%%%%%%%%%%%%%%%%%%%
\begin{frame}
\frametitle{Связный список}
\framesubtitle{Преимущества и недостатки}
\justifying
Связные списки - Преимущества и недостатки
\begin{figure}
    \captionsetup[subfigure]{labelformat=empty}
    \centering
    \subfigure[{ \scriptsize Связные списки - Преимущества и недостатки, Источник - Этот курс}]{\includegraphics[width=135mm, height=50mm]{ll_pros_cons}}
\end{figure}
\end{frame}


%%%%%%%%%%%%%%%%%%%%%%%%%%%%%%%%%%%%%%%%%%%%%%%%%%%%%%%%%%%%%%%%%%%%%%%%%%%%%%%%%%%%%%%%%%%%%%%%%%
\begin{frame}
\frametitle{Простые структуры данных}
\framesubtitle{План лекции}

\begin{enumerate}
  \setcounter{enumi}{-1}
  \item{План лекции}
  \item{Связный список (linked list)}
  \item{\textcolor{blue}{Стек (stack)}}
  \item{Очередь (queue)}
\end{enumerate}
\end{frame}


%%%%%%%%%%%%%%%%%%%%%%%%%%%%%%%%%%%%%%%%%%%%%%%%%%%%%%%%%%%%%%%%%%%%%%%%%%%%%%%%%%%%%%%%%%%%%%%%%%
\begin{frame}
\frametitle{Стек}
\framesubtitle{Стек}
\justifying
\textcolor{red}{Стек (Stack)} - это структура данных, работающая по принципу “последним добавлен – первым возвращен” – Last In First Out (LIFO)

\begin{figure}
    \captionsetup[subfigure]{labelformat=empty}
    \centering
    \subfigure[{ \scriptsize Принцип LIFO, Источник - \href{https://www.geeksforgeeks.org/lifo-principle-in-stack/}{Geeks4Geeks}}]{\includegraphics[width=100mm, height=40mm]{stack}}
\end{figure}
\end{frame}

%%%%%%%%%%%%%%%%%%%%%%%%%%%%%%%%%%%%%%%%%%%%%%%%%%%%%%%%%%%%%%%%%%%%%%%%%%%%%%%%%%%%%%%%%%%%%%%%%%
\begin{frame}
\frametitle{Стек}
\framesubtitle{Реализация на массиве}
\justifying
\textcolor{red}{Стек (Stack)}

\begin{figure}
    \captionsetup[subfigure]{labelformat=empty}
    \centering
    \subfigure[{ \scriptsize Стек, реализация на массиве, Источник - Этот курс}]{\includegraphics[width=100mm, height=35mm]{stack_arrays}}
\end{figure}
\end{frame}

%%%%%%%%%%%%%%%%%%%%%%%%%%%%%%%%%%%%%%%%%%%%%%%%%%%%%%%%%%%%%%%%%%%%%%%%%%%%%%%%%%%%%%%%%%%%%%%%%%
\begin{frame}
\frametitle{Стек}
\framesubtitle{Операции}
Основные операции\newline\newline
\textcolor{blue}{Доступ}
\begin{itemize}
  \item{Получение последнего элемента без удаления}
\end{itemize}
\textcolor{blue}{Вставка}
\begin{itemize}
  \item{В конец стека}
\end{itemize}
\textcolor{blue}{Удаление}
\begin{itemize}
  \item{Получение последнего элемента с удалением}
\end{itemize}
\textcolor{blue}{Дополнительно}
\begin{itemize}
  \item{Пуст ли стек?}
  \item{Заполнен ли стек?}
\end{itemize}
\end{frame}

%%%%%%%%%%%%%%%%%%%%%%%%%%%%%%%%%%%%%%%%%%%%%%%%%%%%%%%%%%%%%%%%%%%%%%%%%%%%%%%%%%%%%%%%%%%%%%%%%%
\begin{frame}
\frametitle{Стек}
\framesubtitle{Реализация на массиве}
\justifying
\textcolor{blue}{Доступ}
\begin{itemize}
  \item{Получение последнего элемента без удаления}
\end{itemize}

\begin{figure}
    \captionsetup[subfigure]{labelformat=empty}
    \centering
    \subfigure[{ \scriptsize Операция peek, Источник - \href{https://www.geeksforgeeks.org/introduction-to-stack-data-structure-and-algorithm-tutorials/}{Geeks4Geeks}}]{\includegraphics[width=65mm, height=45mm]{stack_peek}}
\end{figure}
\end{frame}

%%%%%%%%%%%%%%%%%%%%%%%%%%%%%%%%%%%%%%%%%%%%%%%%%%%%%%%%%%%%%%%%%%%%%%%%%%%%%%%%%%%%%%%%%%%%%%%%%%
\begin{frame}
\frametitle{Стек}
\framesubtitle{Реализация на массиве}
\justifying
\textcolor{blue}{Вставка}
\begin{itemize}
  \item{В конец стека}
\end{itemize}

\begin{figure}
    \captionsetup[subfigure]{labelformat=empty}
    \centering
    \subfigure[{ \scriptsize Операция push, Источник - \href{https://www.geeksforgeeks.org/introduction-to-stack-data-structure-and-algorithm-tutorials/}{Geeks4Geeks}}]{\includegraphics[width=65mm, height=45mm]{stack_push}}
\end{figure}
\end{frame}

%%%%%%%%%%%%%%%%%%%%%%%%%%%%%%%%%%%%%%%%%%%%%%%%%%%%%%%%%%%%%%%%%%%%%%%%%%%%%%%%%%%%%%%%%%%%%%%%%%
\begin{frame}
\frametitle{Стек}
\framesubtitle{Реализация на массиве}
\justifying
\textcolor{blue}{Удаление}
\begin{itemize}
  \item{Получение последнего элемента с удалением}
\end{itemize}

\begin{figure}
    \captionsetup[subfigure]{labelformat=empty}
    \centering
    \subfigure[{ \scriptsize Операция pop, Источник - \href{https://www.geeksforgeeks.org/introduction-to-stack-data-structure-and-algorithm-tutorials/}{Geeks4Geeks}}]{\includegraphics[width=65mm, height=45mm]{stack_pop}}
\end{figure}
\end{frame}

%%%%%%%%%%%%%%%%%%%%%%%%%%%%%%%%%%%%%%%%%%%%%%%%%%%%%%%%%%%%%%%%%%%%%%%%%%%%%%%%%%%%%%%%%%%%%%%%%%
\begin{frame}
\frametitle{Стек}
\framesubtitle{Реализация на массиве}
\justifying
\textcolor{blue}{Дополнительно}
\begin{itemize}
  \item{Пуст ли стек?}
\end{itemize}

\begin{figure}
    \captionsetup[subfigure]{labelformat=empty}
    \centering
    \subfigure[{ \scriptsize Операция isEmpty, Источник - \href{https://www.geeksforgeeks.org/introduction-to-stack-data-structure-and-algorithm-tutorials/}{Geeks4Geeks}}]{\includegraphics[width=65mm, height=45mm]{stack_empty}}
\end{figure}
\end{frame}

%%%%%%%%%%%%%%%%%%%%%%%%%%%%%%%%%%%%%%%%%%%%%%%%%%%%%%%%%%%%%%%%%%%%%%%%%%%%%%%%%%%%%%%%%%%%%%%%%%
\begin{frame}
\frametitle{Стек}
\framesubtitle{Реализация на массиве}
\justifying
\textcolor{blue}{Дополнительно}
\begin{itemize}
  \item{Полон ли стек?}
\end{itemize}

\begin{figure}
    \captionsetup[subfigure]{labelformat=empty}
    \centering
    \subfigure[{ \scriptsize Операция isFull, Источник - \href{https://www.geeksforgeeks.org/introduction-to-stack-data-structure-and-algorithm-tutorials/}{Geeks4Geeks}}]{\includegraphics[width=65mm, height=45mm]{stack_full}}
\end{figure}
\end{frame}

%%%%%%%%%%%%%%%%%%%%%%%%%%%%%%%%%%%%%%%%%%%%%%%%%%%%%%%%%%%%%%%%%%%%%%%%%%%%%%%%%%%%%%%%%%%%%%%%%%
\begin{frame}
\frametitle{Стек}
\framesubtitle{Пример}
\justifying
Пример использования

\begin{figure}
    \captionsetup[subfigure]{labelformat=empty}
    \centering
    \subfigure[{ \scriptsize Операция isFull, Источник - \href{https://www.geeksforgeeks.org/introduction-to-stack-data-structure-and-algorithm-tutorials/}{Geeks4Geeks}}]{\includegraphics[width=120mm, height=42mm]{stack_usage}}
\end{figure}
\end{frame}

%%%%%%%%%%%%%%%%%%%%%%%%%%%%%%%%%%%%%%%%%%%%%%%%%%%%%%%%%%%%%%%%%%%%%%%%%%%%%%%%%%%%%%%%%%%%%%%%%%
\begin{frame}
\frametitle{Стек}
\framesubtitle{Реализация на связном списке}
\justifying
\textcolor{red}{Стек (Stack)}

\begin{figure}
    \captionsetup[subfigure]{labelformat=empty}
    \centering
    \subfigure[{ \scriptsize Стек, реализация на списке, Источник - Этот курс}]{\includegraphics[width=80mm, height=35mm]{stack_ll}}
\end{figure}
\end{frame}

%%%%%%%%%%%%%%%%%%%%%%%%%%%%%%%%%%%%%%%%%%%%%%%%%%%%%%%%%%%%%%%%%%%%%%%%%%%%%%%%%%%%%%%%%%%%%%%%%%
\begin{frame}
\frametitle{Стек}
\framesubtitle{Реализация на связном списке}
\justifying
\textcolor{red}{Стек (Stack)}

\begin{figure}
    \captionsetup[subfigure]{labelformat=empty}
    \centering
    \subfigure[{ \scriptsize Стек, реализация на списке, Источник - \href{https://www.geeksforgeeks.org/introduction-to-stack-data-structure-and-algorithm-tutorials/}{Geeks4Geeks}}]{\includegraphics[width=100mm, height=50mm]{stack_ll_picture}}
\end{figure}
\end{frame}

%%%%%%%%%%%%%%%%%%%%%%%%%%%%%%%%%%%%%%%%%%%%%%%%%%%%%%%%%%%%%%%%%%%%%%%%%%%%%%%%%%%%%%%%%%%%%%%%%%
\begin{frame}
\frametitle{Стек}
\framesubtitle{Сложность}
\justifying
\textcolor{red}{Стек (Stack)}

\begin{figure}
    \captionsetup[subfigure]{labelformat=empty}
    \centering
    \subfigure[{ \scriptsize Стек, сложность, Источник - Этот курс}]{\includegraphics[width=80mm, height=50mm]{stack_complexity_table}}
\end{figure}
\end{frame}

%%%%%%%%%%%%%%%%%%%%%%%%%%%%%%%%%%%%%%%%%%%%%%%%%%%%%%%%%%%%%%%%%%%%%%%%%%%%%%%%%%%%%%%%%%%%%%%%%%
\begin{frame}
\frametitle{Стек}
\framesubtitle{Задачи на стек}
\justifying
Задачи на использование стека\newline\newline
\textcolor{blue}{Определение правильности скобочной последовательности}\newline
Дана строка состоящая из скобок разных видов ( ) [ ] \{ \}. Необходимо сообщить, является ли данная скобочная последовательность правильной.\newline\newline
\textcolor{teal}{()()()[][]\{()\}} – правильная последовательность\newline\newline
\textcolor{teal}{((()))\{\{\}\}[\{\}]} – правильная последовательность\newline\newline
\textcolor{red}{\{[)\}} – неправильная последовательность\newline\newline
\textcolor{red}{\}\{)(} – неправильная последовательность\newline\newline

\end{frame}

%%%%%%%%%%%%%%%%%%%%%%%%%%%%%%%%%%%%%%%%%%%%%%%%%%%%%%%%%%%%%%%%%%%%%%%%%%%%%%%%%%%%%%%%%%%%%%%%%%
\begin{frame}
\frametitle{Простые структуры данных}
\framesubtitle{План лекции}

\begin{enumerate}
  \setcounter{enumi}{-1}
  \item{План лекции}
  \item{Связный список (linked list)}
  \item{Стек (stack)}
  \item{\textcolor{blue}{Очередь (queue)}}
\end{enumerate}
\end{frame}

%%%%%%%%%%%%%%%%%%%%%%%%%%%%%%%%%%%%%%%%%%%%%%%%%%%%%%%%%%%%%%%%%%%%%%%%%%%%%%%%%%%%%%%%%%%%%%%%%%
\begin{frame}
\frametitle{Очередь}
\framesubtitle{Очередь}
\justifying
\textcolor{red}{Очередь (Queue)} - это структура данных, работающая по принципу “последним добавлен – первым возвращен” – Last In First Out (LIFO)

\begin{figure}
    \captionsetup[subfigure]{labelformat=empty}
    \centering
    \subfigure[{ \scriptsize Принцип FIFO, Источник - \href{https://www.geeksforgeeks.org/introduction-and-array-implementation-of-queue/}{Geeks4Geeks}}]{\includegraphics[width=120mm, height=40mm]{queue}}
\end{figure}
\end{frame}

%%%%%%%%%%%%%%%%%%%%%%%%%%%%%%%%%%%%%%%%%%%%%%%%%%%%%%%%%%%%%%%%%%%%%%%%%%%%%%%%%%%%%%%%%%%%%%%%%%
\begin{frame}
\frametitle{Очередь}
\framesubtitle{Очередь}
Основные операции\newline\newline
\textcolor{blue}{Доступ}
\begin{itemize}
  \item{Получение первого элемента без удаления}
  \item{Получение последнего элемента без удаления}
\end{itemize}
\textcolor{blue}{Вставка}
\begin{itemize}
  \item{В конец очереди}
\end{itemize}
\textcolor{blue}{Удаление}
\begin{itemize}
  \item{Удаление из начала очереди}
\end{itemize}
\textcolor{blue}{Дополнительно}
\begin{itemize}
  \item{Пуста ли очередь?}
  \item{Заполнена ли очередь?}
\end{itemize}
\end{frame}

%%%%%%%%%%%%%%%%%%%%%%%%%%%%%%%%%%%%%%%%%%%%%%%%%%%%%%%%%%%%%%%%%%%%%%%%%%%%%%%%%%%%%%%%%%%%%%%%%%
\begin{frame}
\frametitle{Очередь}
\framesubtitle{Очередь}
\justifying
\textcolor{red}{Очередь (Queue)}

\begin{figure}
    \captionsetup[subfigure]{labelformat=empty}
    \centering
    \subfigure[{ \scriptsize Очередь - реализация на массиве, Источник - \href{https://www.geeksforgeeks.org/introduction-and-array-implementation-of-queue/}{Geeks4Geeks}}]{\includegraphics[width=100mm, height=40mm]{queue_array}}
\end{figure}
\end{frame}

%%%%%%%%%%%%%%%%%%%%%%%%%%%%%%%%%%%%%%%%%%%%%%%%%%%%%%%%%%%%%%%%%%%%%%%%%%%%%%%%%%%%%%%%%%%%%%%%%%
\begin{frame}
\frametitle{Очередь}
\framesubtitle{Создание}
\justifying
Операция создания очереди

\begin{figure}
    \captionsetup[subfigure]{labelformat=empty}
    \centering
    \subfigure[{ \scriptsize Создание очереди - реализация на массиве, Источник - \href{https://www.geeksforgeeks.org/introduction-and-array-implementation-of-queue/}{Geeks4Geeks}}]{\includegraphics[width=80mm, height=50mm]{queue_array_create}}
\end{figure}
\end{frame}


%%%%%%%%%%%%%%%%%%%%%%%%%%%%%%%%%%%%%%%%%%%%%%%%%%%%%%%%%%%%%%%%%%%%%%%%%%%%%%%%%%%%%%%%%%%%%%%%%%
\begin{frame}
\frametitle{Очередь}
\framesubtitle{Добавление в очередь}
\justifying
Операция enqueue

\begin{figure}
    \captionsetup[subfigure]{labelformat=empty}
    \centering
    \subfigure[{ \scriptsize Enqueue - реализация на массиве, Источник - \href{https://www.geeksforgeeks.org/introduction-and-array-implementation-of-queue/}{Geeks4Geeks}}]{\includegraphics[width=85mm, height=50mm]{queue_array_enqueue}}
\end{figure}
\end{frame}

%%%%%%%%%%%%%%%%%%%%%%%%%%%%%%%%%%%%%%%%%%%%%%%%%%%%%%%%%%%%%%%%%%%%%%%%%%%%%%%%%%%%%%%%%%%%%%%%%%
\begin{frame}
\frametitle{Очередь}
\framesubtitle{Реализация на связном списке}
\justifying

\begin{figure}
    \captionsetup[subfigure]{labelformat=empty}
    \centering
    \subfigure[{ \scriptsize Очередь на связном списке, Источник - \href{https://www.geeksforgeeks.org/queue-using-linked-list-in-c/}{Geeks4Geeks}}]{\includegraphics[width=100mm, height=50mm]{queue_ll}}
\end{figure}
\end{frame}

%%%%%%%%%%%%%%%%%%%%%%%%%%%%%%%%%%%%%%%%%%%%%%%%%%%%%%%%%%%%%%%%%%%%%%%%%%%%%%%%%%%%%%%%%%%%%%%%%%
\begin{frame}
\frametitle{Очередь}
\framesubtitle{Пример использования}
\justifying

\begin{figure}
    \captionsetup[subfigure]{labelformat=empty}
    \centering
    \subfigure[{ \scriptsize Очередь - пример, Источник - \href{https://www.geeksforgeeks.org/introduction-and-array-implementation-of-queue/}{Geeks4Geeks}}]{\includegraphics[width=120mm, height=50mm]{queue_example}}
\end{figure}
\end{frame}

%%%%%%%%%%%%%%%%%%%%%%%%%%%%%%%%%%%%%%%%%%%%%%%%%%%%%%%%%%%%%%%%%%%%%%%%%%%%%%%%%%%%%%%%%%%%%%%%%%
\begin{frame}
\frametitle{Очередь}
\framesubtitle{Сложность операци (время)}
\justifying

\begin{figure}
    \captionsetup[subfigure]{labelformat=empty}
    \centering
    \subfigure[{ \scriptsize Сложность операций с очередью, Источник - \href{https://www.geeksforgeeks.org/introduction-and-array-implementation-of-queue/}{Geeks4Geeks}}]{\includegraphics[width=65mm, height=55mm]{queue_complexity_table}}
\end{figure}
\end{frame}

\end{document}